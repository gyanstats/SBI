% Options for packages loaded elsewhere
\PassOptionsToPackage{unicode}{hyperref}
\PassOptionsToPackage{hyphens}{url}
%
\documentclass[
]{article}
\usepackage{amsmath,amssymb}
\usepackage{iftex}
\ifPDFTeX
  \usepackage[T1]{fontenc}
  \usepackage[utf8]{inputenc}
  \usepackage{textcomp} % provide euro and other symbols
\else % if luatex or xetex
  \usepackage{unicode-math} % this also loads fontspec
  \defaultfontfeatures{Scale=MatchLowercase}
  \defaultfontfeatures[\rmfamily]{Ligatures=TeX,Scale=1}
\fi
\usepackage{lmodern}
\ifPDFTeX\else
  % xetex/luatex font selection
\fi
% Use upquote if available, for straight quotes in verbatim environments
\IfFileExists{upquote.sty}{\usepackage{upquote}}{}
\IfFileExists{microtype.sty}{% use microtype if available
  \usepackage[]{microtype}
  \UseMicrotypeSet[protrusion]{basicmath} % disable protrusion for tt fonts
}{}
\makeatletter
\@ifundefined{KOMAClassName}{% if non-KOMA class
  \IfFileExists{parskip.sty}{%
    \usepackage{parskip}
  }{% else
    \setlength{\parindent}{0pt}
    \setlength{\parskip}{6pt plus 2pt minus 1pt}}
}{% if KOMA class
  \KOMAoptions{parskip=half}}
\makeatother
\usepackage{xcolor}
\usepackage[margin=1in]{geometry}
\usepackage{color}
\usepackage{fancyvrb}
\newcommand{\VerbBar}{|}
\newcommand{\VERB}{\Verb[commandchars=\\\{\}]}
\DefineVerbatimEnvironment{Highlighting}{Verbatim}{commandchars=\\\{\}}
% Add ',fontsize=\small' for more characters per line
\usepackage{framed}
\definecolor{shadecolor}{RGB}{248,248,248}
\newenvironment{Shaded}{\begin{snugshade}}{\end{snugshade}}
\newcommand{\AlertTok}[1]{\textcolor[rgb]{0.94,0.16,0.16}{#1}}
\newcommand{\AnnotationTok}[1]{\textcolor[rgb]{0.56,0.35,0.01}{\textbf{\textit{#1}}}}
\newcommand{\AttributeTok}[1]{\textcolor[rgb]{0.13,0.29,0.53}{#1}}
\newcommand{\BaseNTok}[1]{\textcolor[rgb]{0.00,0.00,0.81}{#1}}
\newcommand{\BuiltInTok}[1]{#1}
\newcommand{\CharTok}[1]{\textcolor[rgb]{0.31,0.60,0.02}{#1}}
\newcommand{\CommentTok}[1]{\textcolor[rgb]{0.56,0.35,0.01}{\textit{#1}}}
\newcommand{\CommentVarTok}[1]{\textcolor[rgb]{0.56,0.35,0.01}{\textbf{\textit{#1}}}}
\newcommand{\ConstantTok}[1]{\textcolor[rgb]{0.56,0.35,0.01}{#1}}
\newcommand{\ControlFlowTok}[1]{\textcolor[rgb]{0.13,0.29,0.53}{\textbf{#1}}}
\newcommand{\DataTypeTok}[1]{\textcolor[rgb]{0.13,0.29,0.53}{#1}}
\newcommand{\DecValTok}[1]{\textcolor[rgb]{0.00,0.00,0.81}{#1}}
\newcommand{\DocumentationTok}[1]{\textcolor[rgb]{0.56,0.35,0.01}{\textbf{\textit{#1}}}}
\newcommand{\ErrorTok}[1]{\textcolor[rgb]{0.64,0.00,0.00}{\textbf{#1}}}
\newcommand{\ExtensionTok}[1]{#1}
\newcommand{\FloatTok}[1]{\textcolor[rgb]{0.00,0.00,0.81}{#1}}
\newcommand{\FunctionTok}[1]{\textcolor[rgb]{0.13,0.29,0.53}{\textbf{#1}}}
\newcommand{\ImportTok}[1]{#1}
\newcommand{\InformationTok}[1]{\textcolor[rgb]{0.56,0.35,0.01}{\textbf{\textit{#1}}}}
\newcommand{\KeywordTok}[1]{\textcolor[rgb]{0.13,0.29,0.53}{\textbf{#1}}}
\newcommand{\NormalTok}[1]{#1}
\newcommand{\OperatorTok}[1]{\textcolor[rgb]{0.81,0.36,0.00}{\textbf{#1}}}
\newcommand{\OtherTok}[1]{\textcolor[rgb]{0.56,0.35,0.01}{#1}}
\newcommand{\PreprocessorTok}[1]{\textcolor[rgb]{0.56,0.35,0.01}{\textit{#1}}}
\newcommand{\RegionMarkerTok}[1]{#1}
\newcommand{\SpecialCharTok}[1]{\textcolor[rgb]{0.81,0.36,0.00}{\textbf{#1}}}
\newcommand{\SpecialStringTok}[1]{\textcolor[rgb]{0.31,0.60,0.02}{#1}}
\newcommand{\StringTok}[1]{\textcolor[rgb]{0.31,0.60,0.02}{#1}}
\newcommand{\VariableTok}[1]{\textcolor[rgb]{0.00,0.00,0.00}{#1}}
\newcommand{\VerbatimStringTok}[1]{\textcolor[rgb]{0.31,0.60,0.02}{#1}}
\newcommand{\WarningTok}[1]{\textcolor[rgb]{0.56,0.35,0.01}{\textbf{\textit{#1}}}}
\usepackage{graphicx}
\makeatletter
\def\maxwidth{\ifdim\Gin@nat@width>\linewidth\linewidth\else\Gin@nat@width\fi}
\def\maxheight{\ifdim\Gin@nat@height>\textheight\textheight\else\Gin@nat@height\fi}
\makeatother
% Scale images if necessary, so that they will not overflow the page
% margins by default, and it is still possible to overwrite the defaults
% using explicit options in \includegraphics[width, height, ...]{}
\setkeys{Gin}{width=\maxwidth,height=\maxheight,keepaspectratio}
% Set default figure placement to htbp
\makeatletter
\def\fps@figure{htbp}
\makeatother
\setlength{\emergencystretch}{3em} % prevent overfull lines
\providecommand{\tightlist}{%
  \setlength{\itemsep}{0pt}\setlength{\parskip}{0pt}}
\setcounter{secnumdepth}{-\maxdimen} % remove section numbering
\ifLuaTeX
  \usepackage{selnolig}  % disable illegal ligatures
\fi
\IfFileExists{bookmark.sty}{\usepackage{bookmark}}{\usepackage{hyperref}}
\IfFileExists{xurl.sty}{\usepackage{xurl}}{} % add URL line breaks if available
\urlstyle{same}
\hypersetup{
  hidelinks,
  pdfcreator={LaTeX via pandoc}}

\author{}
\date{\vspace{-2.5em}}

\begin{document}

\subsubsection{SM2 / SC2 Project}\label{sm2-sc2-project}

\section{Using ABC/SIR to Model the Spread of Influenza in a Boarding
School}\label{using-abcsir-to-model-the-spread-of-influenza-in-a-boarding-school}

For this group project, we investigated the problem of intractable
likelihoods using Approximate Bayesian Computation. Such simulation
based methods are best applied to this case where we are working with a
model such that we are able to simulate results easily from it, yet have
no analytical form of the likelihood available. This is exactly the case
in epidemiology models, where we are unable to use methods such as
maximum likelihood estimation to estimate the infection parameters given
data.

\subsection{Data-set}\label{data-set}

We begin by importing the dataset we chose for this project: the
\texttt{bsflu} dataset from the package \texttt{pomp}. This dataset
records a 1978 Influenza outbreak in a boy's boarding school.

\begin{Shaded}
\begin{Highlighting}[]
\FunctionTok{library}\NormalTok{(pomp)}
\FunctionTok{library}\NormalTok{(Rcpp)}
\FunctionTok{library}\NormalTok{(sitmo)}
\FunctionTok{library}\NormalTok{(ggplot2)}
\FunctionTok{data}\NormalTok{(bsflu)}
\end{Highlighting}
\end{Shaded}

The dataset tallies infection information over a period of 14 days, in a
boarding school of 763 students.

\begin{Shaded}
\begin{Highlighting}[]
\FunctionTok{head}\NormalTok{(bsflu)}
\end{Highlighting}
\end{Shaded}

\begin{verbatim}
##         date   B  C day
## 1 1978-01-22   1  0   1
## 2 1978-01-23   6  0   2
## 3 1978-01-24  26  0   3
## 4 1978-01-25  73  1   4
## 5 1978-01-26 222  8   5
## 6 1978-01-27 293 16   6
\end{verbatim}

The column \texttt{B} contains the number of students who are bedridden
with the flu on a given day (i.e.~classes as `infected'). C contains the
number of students who are \emph{convalescent} (i.e.~not infected but
yet unable to return to class).

\begin{Shaded}
\begin{Highlighting}[]
\FunctionTok{library}\NormalTok{(tidyr)}
\FunctionTok{library}\NormalTok{(ggplot2)}


\NormalTok{bsflu }\SpecialCharTok{|\textgreater{}}
    \FunctionTok{gather}\NormalTok{(variable,value,}\SpecialCharTok{{-}}\NormalTok{date,}\SpecialCharTok{{-}}\NormalTok{day) }\SpecialCharTok{|\textgreater{}}
    \FunctionTok{ggplot}\NormalTok{(}\FunctionTok{aes}\NormalTok{(}\AttributeTok{x=}\NormalTok{date,}\AttributeTok{y=}\NormalTok{value,}\AttributeTok{color=}\NormalTok{variable))}\SpecialCharTok{+}
    \FunctionTok{geom\_line}\NormalTok{()}\SpecialCharTok{+}
    \FunctionTok{geom\_point}\NormalTok{()}\SpecialCharTok{+}
    \FunctionTok{labs}\NormalTok{(}\AttributeTok{y=}\StringTok{"Number of Boys"}\NormalTok{,}\AttributeTok{x=}\StringTok{"Date"}\NormalTok{,}\AttributeTok{title=}\StringTok{"Boarding school infuenza outbreak (22nd Jan {-} 4th Feb)"}\NormalTok{)}\SpecialCharTok{+}
  \FunctionTok{scale\_x\_date}\NormalTok{(}\AttributeTok{date\_minor\_breaks =} \StringTok{"1 day"}\NormalTok{,}\AttributeTok{date\_labels =} \StringTok{"\%b \%d"}\NormalTok{)}\SpecialCharTok{+}
    \FunctionTok{theme\_light}\NormalTok{()}
\end{Highlighting}
\end{Shaded}

\includegraphics{Project_files/figure-latex/unnamed-chunk-3-1.pdf}

\subsection{Model}\label{model}

In order to model disease data, we will use the well-studied SIR model.
This model models the number of people in three states: Susceptible,
Infected, and Recovered. The model is defined by the following system of
differential equations: \[
\begin{align*}
\frac{dS}{dt} &= -\beta S I \\
\frac{dI}{dt} &= \beta S I - \gamma I \\
\frac{dR}{dt} &= \gamma I
\end{align*}
\]

Where \(S\) is the proportion of susceptible individuals, \(I\) is the
proportion of infected individuals, and \(R\) is the proportion of
recovered individuals. \(\beta\) is the rate of infection, and
\(\gamma\) is the rate of recovery.

With such a definition, we can translate the column \texttt{B} in the
\texttt{bsflu} data directly to the variable \(I\) simply by dividing
\texttt{B} by the total number of students (\(N=763\)). Unfortunately,
the column \texttt{C} has no analogy in the model, as it acts as a
confusing `between recovery' state that cannot be grouped in with either
\(I\) or \(R\). Therefore going forward, we will primarily be using the
column \texttt{B} as the observed data in our SIR model estimate.

Approximate Bayesian computation is a simulation based approach, and
will require many individual computations. In the interest of speed
therefore, we will implement the SIR model in C++, then use RCPP to call
the C++ code from R.

\begin{Shaded}
\begin{Highlighting}[]
\FunctionTok{sourceCpp}\NormalTok{(}\AttributeTok{code =} \StringTok{"}
\StringTok{  \#include \textless{}Rcpp.h\textgreater{}}
\StringTok{  using namespace Rcpp;}

\StringTok{  // [[Rcpp::export]]}
\StringTok{  NumericVector SIR(NumericVector s, NumericVector i, NumericVector r, double s0, double i0, double r0, double beta, double gamma) \{}
\StringTok{    s[0] = s0;}
\StringTok{    i[0] = i0;}
\StringTok{    r[0] = r0;}

\StringTok{    for (int t = 1; t \textless{} s.size(); t++) \{}
\StringTok{      s[t] = s[t{-}1] {-} beta * s[t{-}1] * i[t{-}1];}
\StringTok{      i[t] = i[t{-}1] + beta * s[t{-}1] * i[t{-}1] {-} gamma * i[t{-}1];}
\StringTok{      r[t] = r[t{-}1] + gamma * i[t{-}1];}
\StringTok{    \}}

\StringTok{    return i;}
\StringTok{  \}}
\StringTok{"}\NormalTok{)}
\end{Highlighting}
\end{Shaded}

Below is a demonstration of the SIR model above, with \(\beta = 0.8\)
and \(\gamma = 0.2\).

\begin{Shaded}
\begin{Highlighting}[]
\NormalTok{s }\OtherTok{\textless{}{-}} \FunctionTok{numeric}\NormalTok{(}\DecValTok{20}\NormalTok{)}
\NormalTok{i }\OtherTok{\textless{}{-}} \FunctionTok{numeric}\NormalTok{(}\DecValTok{20}\NormalTok{)}
\NormalTok{r }\OtherTok{\textless{}{-}} \FunctionTok{numeric}\NormalTok{(}\DecValTok{20}\NormalTok{)}

\NormalTok{sir }\OtherTok{\textless{}{-}} \FunctionTok{SIR}\NormalTok{(s, i, r, }\DecValTok{1} \SpecialCharTok{{-}} \DecValTok{1}\SpecialCharTok{/}\DecValTok{763}\NormalTok{, }\DecValTok{1}\SpecialCharTok{/}\DecValTok{763}\NormalTok{, }\DecValTok{0}\NormalTok{, }\FloatTok{0.8}\NormalTok{, }\FloatTok{0.2}\NormalTok{)}

\CommentTok{\# convert SIR to a data frame}
\NormalTok{sir }\OtherTok{\textless{}{-}} \FunctionTok{data.frame}\NormalTok{(}\AttributeTok{day =} \DecValTok{1}\SpecialCharTok{:}\DecValTok{20}\NormalTok{, }\AttributeTok{sir =}\NormalTok{ sir)}
\FunctionTok{print}\NormalTok{(sir)}
\end{Highlighting}
\end{Shaded}

\begin{verbatim}
##    day         sir
## 1    1 0.001310616
## 2    2 0.002095611
## 3    3 0.003349026
## 4    4 0.005347643
## 5    5 0.008527571
## 6    6 0.013569424
## 7    7 0.021518985
## 8    8 0.033942171
## 9    9 0.053083221
## 10  10 0.081917372
## 11  11 0.123828512
## 12  12 0.181407697
## 13  13 0.253810306
## 14  14 0.333041786
## 15  15 0.402372183
## 16  16 0.442376936
## 17  17 0.443721280
## 18  18 0.413185769
## 19  19 0.365510796
## 20  20 0.313113502
\end{verbatim}

\begin{Shaded}
\begin{Highlighting}[]
\FunctionTok{ggplot}\NormalTok{(}\FunctionTok{aes}\NormalTok{(}\AttributeTok{x=}\NormalTok{day,}\AttributeTok{y=}\NormalTok{sir), }\AttributeTok{data =}\NormalTok{ sir)}\SpecialCharTok{+}
  \FunctionTok{geom\_line}\NormalTok{(}\AttributeTok{color=}\StringTok{"red"}\NormalTok{)}\SpecialCharTok{+}
  \FunctionTok{geom\_point}\NormalTok{(}\AttributeTok{color=}\StringTok{"red"}\NormalTok{)}\SpecialCharTok{+}
  \FunctionTok{labs}\NormalTok{(}\AttributeTok{y=}\StringTok{"Proportion of Infected"}\NormalTok{,}\AttributeTok{x=}\StringTok{"Day"}\NormalTok{,}\AttributeTok{title=}\StringTok{"SIR Model with beta=0.8, gamma=0.2"}\NormalTok{)}\SpecialCharTok{+}\FunctionTok{theme\_light}\NormalTok{()}
\end{Highlighting}
\end{Shaded}

\includegraphics{Project_files/figure-latex/unnamed-chunk-5-1.pdf}

\subsection{ABC Definition}\label{abc-definition}

Consider first the general case of intractable likelihood: having a
model \(f(.|\theta)\) with intractable likelihood \(l(\theta|.)\) and
parameter \(\theta\).

For ABC, we first repeatedly generate samples of \(\theta\) from a prior
distribution \(\theta \sim \pi(.)\). Then for each generated value of
\(\theta\), we input this into our effectively `black box' model to get
simulated data \(\tilde{y}(\theta) \sim f(.|\theta)\).

We then need to define some distance metric \(D\) between the observed
data \(y\) and simulated data \(\tilde{y}(\theta)\), only accepting the
proposed \(\theta\) if this distance falls below a defined tolerance
value \(\epsilon\).

Even a simple rejection sampling algorithm such as this can be shown to
produce samples \(\{\theta_1,...,\theta_M\}\) (\(M\) being the number of
accepted values) that are samples from the joint distribution:\[
\begin{align*}
    \pi_{\epsilon}(\theta,\tilde{y}|y) = \frac{\pi(\theta)f(\tilde{y}|\theta)\mathbb{I}(\tilde{y}\in A)}{\int_A\int_\Theta\pi(\theta)f(\tilde{y}|\theta) d \tilde{y} d \theta}
\end{align*}\] Where \(\mathbb{I}\) is the indicator function,
\(\Theta\) is the support of \(\theta\), and \(A\) is the acceptance
region defined by \(D\), \(y\), and \(\epsilon\). Then given a suitable
choice of tolerance value, this can produce an approximation to the
posterior distribution of \(\theta\) {[}1{]}.\[
\begin{align*}
    \pi_{\epsilon}(\theta|y) = \int_A\pi_{\epsilon}(\theta,\tilde{y}|y)d\tilde{y} \approx \pi(\theta|y)
\end{align*}\]

Clearly here much of the resulting estimate relies on our choice of
tolerance parameter \(\epsilon\) and distance metric \(D\), both of
which will be looked at later. Something else to consider is that in
practice the distance metric is applied to a \emph{summary statistic} of
the data rather than the raw data, in order to reduce dimensionality.
This can be anything from the mean \(\bar{y}\) and empirical quantiles,
to more complex statistics such as kernels or auxiliary parameters.
These methods will be looked at near the tail end of our investigation.

\subsection{ABC Implementation}\label{abc-implementation}

For the distance metric within ABC, we will need to compare the
simulated data to the observed data. However as noted in the Data-set
section, only the column B can be used. Hence we will compare the B
column of the dataset to the number of infected individuals I in our SIR
model. Similar to the approach of {[}2{]}, we will use the mean squared
error between the proportions of infected individuals in the observed
data and the simulated data as our distance metric, and compare it to
the other distance metric used; the absolute error between the
proportion of infected individuals on the final day of the observed data
and the simulated data.

The following code implements the ABC/SIR algorithm in C++.

\begin{Shaded}
\begin{Highlighting}[]
\FunctionTok{sourceCpp}\NormalTok{(}\AttributeTok{code=}\StringTok{\textquotesingle{}}
\StringTok{\#include \textless{}Rcpp.h\textgreater{}}
\StringTok{\#include \textless{}RcppParallel.h\textgreater{}}
\StringTok{\#include \textless{}omp.h\textgreater{}}
\StringTok{\#include \textless{}sitmo.h\textgreater{}}
\StringTok{\#include \textless{}cmath\textgreater{}}
\StringTok{using namespace Rcpp;}

\StringTok{// [[Rcpp::depends(RcppParallel)]]}
\StringTok{// [[Rcpp::depends(sitmo)]]}
\StringTok{// [[Rcpp::plugins(openmp)]]}

\StringTok{// Function to simulate data from SIR model}

\StringTok{// [[Rcpp::export]]}
\StringTok{double unif\_sitmo(int seed) \{}
\StringTok{  uint32\_t coreseed = static\_cast\textless{}uint32\_t\textgreater{}(seed);}
\StringTok{  sitmo::prng eng(coreseed);}
\StringTok{  double mx = sitmo::prng::max();}
\StringTok{  double x = eng() / mx;}
\StringTok{  return x;}
\StringTok{\}}

\StringTok{double max\_double(double* x)\{}

\StringTok{  double m = 0;}
\StringTok{  for (int i=0; i\textless{}20; i++) \{}
\StringTok{    if(x[i] \textgreater{} m)\{}
\StringTok{      m = x[i];}
\StringTok{    \}}
\StringTok{  \}}
\StringTok{  return m;}
\StringTok{\}}

\StringTok{double calc\_dist\_serial(double* x\_sim, NumericVector x) \{}
\StringTok{  double total;}
\StringTok{  for (int i=0; i\textless{}20; i++) \{}
\StringTok{    total += pow(x[i]{-}x\_sim[i],2);}
\StringTok{  \}}
\StringTok{  return total;}
\StringTok{\}}

\StringTok{double mean\_std\_dist(double* x\_sim, NumericVector x, NumericVector w) \{}
\StringTok{  double mean;}
\StringTok{  double sd;}
\StringTok{  double s;}
\StringTok{  for (int i=0;i \textless{} 20; i++)\{}
\StringTok{    s += x\_sim[i];}
\StringTok{  \}}
\StringTok{  for (int i=0;i \textless{} 20; i++)\{}
\StringTok{    mean += (i+1)*x\_sim[i]/s;}
\StringTok{  \}}
\StringTok{  for (int i=0;i \textless{} 20; i++)\{}
\StringTok{    sd += pow(i+1{-}mean,2)*x\_sim[i]/s;}
\StringTok{  \}}

\StringTok{  sd = sqrt(sd);}
\StringTok{  }
\StringTok{  double total = w[0]*pow(mean{-}13.14,2)/pow(13.14,2)+w[1]*pow(sd{-}2.19,2)/pow(2.19,2)+w[2]*pow(max\_double(x\_sim){-}max(x),2)/(pow(max(x),2));}
\StringTok{  }
\StringTok{  return total;}

\StringTok{\}}

\StringTok{// [[Rcpp::export]]}
\StringTok{NumericMatrix ABC(int n, double eps, int p, NumericVector x, int ncores, int metric, NumericVector w)}
\StringTok{\{}
\StringTok{  }
\StringTok{  NumericMatrix accepted\_samples(n, p);}
\StringTok{  int count = 0;}
\StringTok{  double dist;}

\StringTok{  \#pragma omp parallel num\_threads(ncores)}
\StringTok{  \{}
\StringTok{    double theta\_sim[2];}
\StringTok{    \#pragma omp for}
\StringTok{    for (int i=0; i\textless{}n; i++) \{}

\StringTok{      \#pragma omp critical}
\StringTok{      \{}
\StringTok{      theta\_sim[0] = 3*unif\_sitmo(i);}
\StringTok{      theta\_sim[1] = unif\_sitmo(i+n);}
\StringTok{      \}}
\StringTok{      }
\StringTok{      // NumericVector I = SIR(762.0/763.0, 1.0/763.0, 0.0, theta\_sim[1], theta\_sim[2]);}

\StringTok{      double S[20];}
\StringTok{      double I[20];}
\StringTok{      double R[20];}

\StringTok{      S[0] = 762.0/763.0;}
\StringTok{      I[0] = 1.0/763.0;}
\StringTok{      R[0] = 0.0;}

\StringTok{      for (int t = 1; t \textless{} 20; t++) \{}
\StringTok{        S[t] = S[t{-}1] {-} theta\_sim[0] * S[t{-}1] * I[t{-}1];}
\StringTok{        I[t] = I[t{-}1] + theta\_sim[0] * S[t{-}1] * I[t{-}1] {-} theta\_sim[1] * I[t{-}1];}
\StringTok{        R[t] = R[t{-}1] + theta\_sim[1] * I[t{-}1];}
\StringTok{      \}}

\StringTok{      \#pragma omp critical}
\StringTok{      \{}
\StringTok{      if (metric == 1) \{}
\StringTok{        dist = calc\_dist\_serial(I, x);}
\StringTok{      \} else if (metric == 2) \{}
\StringTok{        dist = pow(I[19]{-}x[19], 2);}
\StringTok{      \} else if (metric == 3) \{}
\StringTok{        dist = mean\_std\_dist(I, x, w);}
\StringTok{      \}}
\StringTok{      if (dist \textless{} eps) \{}
\StringTok{        accepted\_samples(i, 0) = theta\_sim[0];}
\StringTok{        accepted\_samples(i, 1) = theta\_sim[1];}
\StringTok{        count++;}
\StringTok{      \}}
\StringTok{      \}}
\StringTok{    \}}
\StringTok{  \}}
\StringTok{  }
\StringTok{  std::cout \textless{}\textless{} "Acceptance rate: " \textless{}\textless{} (double)count / n \textless{}\textless{} std::endl;}
\StringTok{  return accepted\_samples;}
\StringTok{  }
\StringTok{\}}

\StringTok{\textquotesingle{}}\NormalTok{)}
\end{Highlighting}
\end{Shaded}

The function \texttt{ABC} takes in the number of samples \texttt{n}, the
tolerance \texttt{eps}, the number of parameters \texttt{p}, the
observed data \texttt{x}, and the number of cores to use
\texttt{ncores}. It then returns a matrix of accepted samples. It is
important to note that we should normalise the \texttt{x} vector before
inputting it into the ABC function, as the SIR model uses proportions
rather than raw numbers.

\subsection{Comparison of Metrics}\label{comparison-of-metrics}

As an example, we can run the ABC algorithm with \(10^7\) samples, a
tolerance of 0.8, and 2 parameters, using the mean squared error
initially as the distance metric.

\begin{Shaded}
\begin{Highlighting}[]
\NormalTok{x }\OtherTok{\textless{}{-}} \FunctionTok{c}\NormalTok{(}\FunctionTok{rep}\NormalTok{(}\DecValTok{1}\SpecialCharTok{/}\DecValTok{763}\NormalTok{,}\DecValTok{5}\NormalTok{), }\DecValTok{2}\SpecialCharTok{/}\DecValTok{763}\NormalTok{, bsflu}\SpecialCharTok{$}\NormalTok{B}\SpecialCharTok{/}\DecValTok{763}\NormalTok{)}
\FunctionTok{length}\NormalTok{(x)}
\end{Highlighting}
\end{Shaded}

\begin{verbatim}
## [1] 20
\end{verbatim}

\begin{Shaded}
\begin{Highlighting}[]
\NormalTok{accepted\_samples }\OtherTok{\textless{}{-}} \FunctionTok{ABC}\NormalTok{(}\FloatTok{1e6}\NormalTok{, .}\DecValTok{8}\NormalTok{, }\DecValTok{2}\NormalTok{, x, }\DecValTok{4}\NormalTok{, }\DecValTok{1}\NormalTok{,}\ConstantTok{NA}\NormalTok{)}

\CommentTok{\# remove zeros}
\NormalTok{accepted\_samples }\OtherTok{\textless{}{-}}\NormalTok{ accepted\_samples[}\SpecialCharTok{!}\FunctionTok{rowSums}\NormalTok{(accepted\_samples}\SpecialCharTok{==}\DecValTok{0}\NormalTok{),]}
\end{Highlighting}
\end{Shaded}

We can then plot the marginal posterior distributions of the parameters
\(\beta\) and \(\gamma\).

\begin{Shaded}
\begin{Highlighting}[]
\FunctionTok{library}\NormalTok{(cowplot)}
\CommentTok{\# plot the density histograms}
\NormalTok{example\_beta\_plot}\OtherTok{\textless{}{-}}\FunctionTok{ggplot}\NormalTok{(}\FunctionTok{aes}\NormalTok{(}\AttributeTok{x=}\NormalTok{accepted\_samples[,}\DecValTok{1}\NormalTok{]), }\AttributeTok{data =} \FunctionTok{as.data.frame}\NormalTok{(accepted\_samples))}\SpecialCharTok{+}
  \FunctionTok{geom\_density}\NormalTok{(}\AttributeTok{fill=}\StringTok{"lightblue"}\NormalTok{,}\AttributeTok{color=}\StringTok{"blue"}\NormalTok{)}\SpecialCharTok{+}
  \FunctionTok{xlim}\NormalTok{(}\FunctionTok{c}\NormalTok{(}\DecValTok{0}\NormalTok{,}\DecValTok{3}\NormalTok{))}\SpecialCharTok{+}
  \FunctionTok{labs}\NormalTok{(}\AttributeTok{x=}\StringTok{"Beta"}\NormalTok{,}\AttributeTok{y=}\StringTok{"Frequency"}\NormalTok{,}\AttributeTok{title=}\FunctionTok{expression}\NormalTok{(}\FunctionTok{paste}\NormalTok{(}\StringTok{"Marginal Posterior Distribution of "}\NormalTok{, beta)))}\SpecialCharTok{+}\FunctionTok{theme\_light}\NormalTok{()}

\NormalTok{example\_gamma\_plot}\OtherTok{\textless{}{-}}\FunctionTok{ggplot}\NormalTok{(}\FunctionTok{aes}\NormalTok{(}\AttributeTok{x=}\NormalTok{accepted\_samples[,}\DecValTok{2}\NormalTok{]), }\AttributeTok{data =} \FunctionTok{as.data.frame}\NormalTok{(accepted\_samples))}\SpecialCharTok{+}
  \FunctionTok{geom\_density}\NormalTok{(}\AttributeTok{fill=}\StringTok{"pink"}\NormalTok{,}\AttributeTok{color=}\StringTok{"red"}\NormalTok{)}\SpecialCharTok{+}
  \FunctionTok{xlim}\NormalTok{(}\FunctionTok{c}\NormalTok{(}\DecValTok{0}\NormalTok{,}\DecValTok{1}\NormalTok{))}\SpecialCharTok{+}
  \FunctionTok{labs}\NormalTok{(}\AttributeTok{x=}\StringTok{"Gamma"}\NormalTok{,}\AttributeTok{y=}\StringTok{"Density"}\NormalTok{,}\AttributeTok{title=}\FunctionTok{expression}\NormalTok{(}\FunctionTok{paste}\NormalTok{(}\StringTok{"Marginal Posterior Distribution of "}\NormalTok{, gamma)))}\SpecialCharTok{+}\FunctionTok{theme\_light}\NormalTok{()}

\FunctionTok{plot\_grid}\NormalTok{(example\_beta\_plot, example\_gamma\_plot)}
\end{Highlighting}
\end{Shaded}

\includegraphics{Project_files/figure-latex/unnamed-chunk-8-1.pdf}

Now we can compare these posterior distributions to the distributions
obtained when we use the absolute error between the proportion of
infected individuals on the final day of the observed data and the
simulated data as the distance metric.

\begin{Shaded}
\begin{Highlighting}[]
\NormalTok{accepted\_samples2 }\OtherTok{\textless{}{-}} \FunctionTok{ABC}\NormalTok{(}\FloatTok{1e6}\NormalTok{, }\FloatTok{0.001}\NormalTok{, }\DecValTok{2}\NormalTok{, x, }\DecValTok{4}\NormalTok{, }\DecValTok{2}\NormalTok{,}\ConstantTok{NA}\NormalTok{)}
\NormalTok{accepted\_samples2 }\OtherTok{\textless{}{-}}\NormalTok{ accepted\_samples2[}\SpecialCharTok{!}\FunctionTok{rowSums}\NormalTok{(accepted\_samples2}\SpecialCharTok{==}\DecValTok{0}\NormalTok{),]}
\end{Highlighting}
\end{Shaded}

\begin{Shaded}
\begin{Highlighting}[]
\NormalTok{abs\_beta\_plot}\OtherTok{\textless{}{-}}\FunctionTok{ggplot}\NormalTok{(}\FunctionTok{aes}\NormalTok{(}\AttributeTok{x=}\NormalTok{accepted\_samples2[,}\DecValTok{1}\NormalTok{]), }\AttributeTok{data =} \FunctionTok{as.data.frame}\NormalTok{(accepted\_samples2))}\SpecialCharTok{+}
  \FunctionTok{geom\_density}\NormalTok{(}\AttributeTok{fill=}\StringTok{"lightblue"}\NormalTok{,}\AttributeTok{color=}\StringTok{"blue"}\NormalTok{)}\SpecialCharTok{+}
  \FunctionTok{labs}\NormalTok{(}\AttributeTok{x=}\StringTok{"Beta"}\NormalTok{,}\AttributeTok{y=}\StringTok{"Frequency"}\NormalTok{,}\AttributeTok{title=}\FunctionTok{expression}\NormalTok{(}\FunctionTok{paste}\NormalTok{(}\StringTok{"Marginal Posterior Distribution of "}\NormalTok{, beta)))}\SpecialCharTok{+}\FunctionTok{theme\_light}\NormalTok{()}

\NormalTok{abs\_gamma\_plot}\OtherTok{\textless{}{-}}\FunctionTok{ggplot}\NormalTok{(}\FunctionTok{aes}\NormalTok{(}\AttributeTok{x=}\NormalTok{accepted\_samples2[,}\DecValTok{2}\NormalTok{]), }\AttributeTok{data =} \FunctionTok{as.data.frame}\NormalTok{(accepted\_samples2))}\SpecialCharTok{+}
  \FunctionTok{geom\_density}\NormalTok{(}\AttributeTok{fill=}\StringTok{"pink"}\NormalTok{,}\AttributeTok{color=}\StringTok{"red"}\NormalTok{)}\SpecialCharTok{+}
  \FunctionTok{labs}\NormalTok{(}\AttributeTok{x=}\StringTok{"Gamma"}\NormalTok{,}\AttributeTok{y=}\StringTok{"Density"}\NormalTok{,}\AttributeTok{title=}\FunctionTok{expression}\NormalTok{(}\FunctionTok{paste}\NormalTok{(}\StringTok{"Marginal Posterior Distribution of "}\NormalTok{, gamma)))}\SpecialCharTok{+}\FunctionTok{theme\_light}\NormalTok{()}

\FunctionTok{plot\_grid}\NormalTok{(abs\_beta\_plot,abs\_gamma\_plot)}
\end{Highlighting}
\end{Shaded}

\includegraphics{Project_files/figure-latex/unnamed-chunk-10-1.pdf}

\subsection{Acceptance Rates}\label{acceptance-rates}

We can plot the acceptance rates for the two distance metrics as a
function of the tolerance value.

\begin{Shaded}
\begin{Highlighting}[]
\NormalTok{eps }\OtherTok{\textless{}{-}} \FunctionTok{seq}\NormalTok{(}\FloatTok{0.4}\NormalTok{, }\DecValTok{1}\NormalTok{, }\AttributeTok{length.out=}\DecValTok{20}\NormalTok{)}
\NormalTok{acceptance\_rates }\OtherTok{\textless{}{-}} \FunctionTok{numeric}\NormalTok{(}\FunctionTok{length}\NormalTok{(eps))}
\ControlFlowTok{for}\NormalTok{ (i }\ControlFlowTok{in} \DecValTok{1}\SpecialCharTok{:}\FunctionTok{length}\NormalTok{(eps)) \{}
\NormalTok{  accepted\_samples }\OtherTok{\textless{}{-}} \FunctionTok{ABC}\NormalTok{(}\FloatTok{1e6}\NormalTok{, eps[i], }\DecValTok{2}\NormalTok{, x, }\DecValTok{4}\NormalTok{, }\DecValTok{1}\NormalTok{,}\ConstantTok{NA}\NormalTok{)}
\NormalTok{  acceptance\_rates[i] }\OtherTok{\textless{}{-}} \FunctionTok{sum}\NormalTok{(accepted\_samples[,}\DecValTok{1}\NormalTok{] }\SpecialCharTok{!=} \DecValTok{0}\NormalTok{) }\SpecialCharTok{/} \FunctionTok{nrow}\NormalTok{(accepted\_samples)}
\NormalTok{\}}
\end{Highlighting}
\end{Shaded}

\begin{Shaded}
\begin{Highlighting}[]
\NormalTok{eps2 }\OtherTok{\textless{}{-}} \FunctionTok{seq}\NormalTok{(}\DecValTok{0}\NormalTok{, }\FloatTok{0.1}\NormalTok{, }\AttributeTok{length.out=}\DecValTok{20}\NormalTok{)}
\NormalTok{acceptance\_rates2 }\OtherTok{\textless{}{-}} \FunctionTok{numeric}\NormalTok{(}\FunctionTok{length}\NormalTok{(eps2))}
\ControlFlowTok{for}\NormalTok{ (i }\ControlFlowTok{in} \DecValTok{1}\SpecialCharTok{:}\FunctionTok{length}\NormalTok{(eps2)) \{}
\NormalTok{  accepted\_samples2 }\OtherTok{\textless{}{-}} \FunctionTok{ABC}\NormalTok{(}\FloatTok{1e6}\NormalTok{, eps2[i], }\DecValTok{2}\NormalTok{, x, }\DecValTok{4}\NormalTok{, }\DecValTok{2}\NormalTok{,}\ConstantTok{NA}\NormalTok{)}
\NormalTok{    acceptance\_rates2[i] }\OtherTok{\textless{}{-}} \FunctionTok{sum}\NormalTok{(accepted\_samples2[,}\DecValTok{1}\NormalTok{] }\SpecialCharTok{!=} \DecValTok{0}\NormalTok{) }\SpecialCharTok{/} \FunctionTok{nrow}\NormalTok{(accepted\_samples)}
\NormalTok{\}}
\end{Highlighting}
\end{Shaded}

\begin{Shaded}
\begin{Highlighting}[]
\NormalTok{msedata}\OtherTok{\textless{}{-}}\FunctionTok{data.frame}\NormalTok{(}\AttributeTok{Tolerance=}\NormalTok{eps,}\AttributeTok{Acceptance\_Rate=}\NormalTok{acceptance\_rates)}
\NormalTok{absdata}\OtherTok{\textless{}{-}}\FunctionTok{data.frame}\NormalTok{(}\AttributeTok{Tolerance=}\NormalTok{eps2, }\AttributeTok{Acceptance\_Rate=}\NormalTok{acceptance\_rates2)}
\NormalTok{mse\_plot}\OtherTok{\textless{}{-}}\FunctionTok{ggplot}\NormalTok{(msedata,}\FunctionTok{aes}\NormalTok{(}\AttributeTok{x=}\NormalTok{Tolerance,}\AttributeTok{y=}\NormalTok{Acceptance\_Rate))}\SpecialCharTok{+}
  \FunctionTok{geom\_line}\NormalTok{(}\AttributeTok{col=}\StringTok{\textquotesingle{}darkgreen\textquotesingle{}}\NormalTok{) }\SpecialCharTok{+} \FunctionTok{geom\_point}\NormalTok{(}\AttributeTok{col=}\StringTok{\textquotesingle{}darkgreen\textquotesingle{}}\NormalTok{) }\SpecialCharTok{+} \FunctionTok{theme\_light}\NormalTok{() }\SpecialCharTok{+} \FunctionTok{ggtitle}\NormalTok{(}\StringTok{"Mean Squred Error"}\NormalTok{)}
\NormalTok{abs\_plot}\OtherTok{\textless{}{-}}\FunctionTok{ggplot}\NormalTok{(absdata,}\FunctionTok{aes}\NormalTok{(}\AttributeTok{x=}\NormalTok{Tolerance,}\AttributeTok{y=}\NormalTok{Acceptance\_Rate))}\SpecialCharTok{+}
  \FunctionTok{geom\_line}\NormalTok{(}\AttributeTok{col=}\StringTok{\textquotesingle{}purple\textquotesingle{}}\NormalTok{) }\SpecialCharTok{+} \FunctionTok{geom\_point}\NormalTok{(}\AttributeTok{col=}\StringTok{\textquotesingle{}purple\textquotesingle{}}\NormalTok{) }\SpecialCharTok{+} \FunctionTok{theme\_light}\NormalTok{() }\SpecialCharTok{+} \FunctionTok{ggtitle}\NormalTok{(}\StringTok{"Absolute Value"}\NormalTok{)}
\FunctionTok{plot\_grid}\NormalTok{(mse\_plot,abs\_plot)}
\end{Highlighting}
\end{Shaded}

\includegraphics{Project_files/figure-latex/unnamed-chunk-13-1.pdf}

We see that there is a sharp transition in the acceptance rate in the
mean squared error plot. Below we plot the marginal posterior
distribution of \(\beta\) and \(\gamma\) at a tolerance before and after
the transition.

\begin{Shaded}
\begin{Highlighting}[]
\NormalTok{accepted\_samples\_before }\OtherTok{\textless{}{-}} \FunctionTok{ABC}\NormalTok{(}\FloatTok{1e6}\NormalTok{, }\FloatTok{0.55}\NormalTok{, }\DecValTok{2}\NormalTok{, x, }\DecValTok{4}\NormalTok{, }\DecValTok{1}\NormalTok{,}\ConstantTok{NA}\NormalTok{)}
\NormalTok{accepted\_samples\_before }\OtherTok{\textless{}{-}}\NormalTok{ accepted\_samples\_before[}\SpecialCharTok{!}\FunctionTok{rowSums}\NormalTok{(accepted\_samples\_before}\SpecialCharTok{==}\DecValTok{0}\NormalTok{),]}
\NormalTok{accepted\_samples\_after }\OtherTok{\textless{}{-}} \FunctionTok{ABC}\NormalTok{(}\FloatTok{1e6}\NormalTok{, }\FloatTok{0.6}\NormalTok{, }\DecValTok{2}\NormalTok{, x, }\DecValTok{4}\NormalTok{, }\DecValTok{1}\NormalTok{,}\ConstantTok{NA}\NormalTok{)}
\NormalTok{accepted\_samples\_after }\OtherTok{\textless{}{-}}\NormalTok{ accepted\_samples\_after[}\SpecialCharTok{!}\FunctionTok{rowSums}\NormalTok{(accepted\_samples\_after}\SpecialCharTok{==}\DecValTok{0}\NormalTok{),]}
\end{Highlighting}
\end{Shaded}

\begin{Shaded}
\begin{Highlighting}[]
\NormalTok{before\_beta\_plot}\OtherTok{\textless{}{-}}\FunctionTok{ggplot}\NormalTok{(}\FunctionTok{aes}\NormalTok{(}\AttributeTok{x=}\NormalTok{accepted\_samples\_before[,}\DecValTok{1}\NormalTok{]), }\AttributeTok{data =} \FunctionTok{as.data.frame}\NormalTok{(accepted\_samples\_before))}\SpecialCharTok{+}
  \FunctionTok{geom\_density}\NormalTok{(}\AttributeTok{fill=}\StringTok{"lightblue"}\NormalTok{,}\AttributeTok{color=}\StringTok{"blue"}\NormalTok{)}\SpecialCharTok{+}
  \FunctionTok{xlim}\NormalTok{(}\FunctionTok{c}\NormalTok{(}\DecValTok{0}\NormalTok{,}\DecValTok{3}\NormalTok{))}\SpecialCharTok{+}
  \FunctionTok{labs}\NormalTok{(}\AttributeTok{x=}\StringTok{"Beta"}\NormalTok{,}\AttributeTok{y=}\StringTok{"Frequency"}\NormalTok{,}\AttributeTok{title=}\FunctionTok{expression}\NormalTok{(}\FunctionTok{paste}\NormalTok{(}\StringTok{"Distribution of "}\NormalTok{, beta,}\StringTok{"  (Before)"}\NormalTok{)))}\SpecialCharTok{+}\FunctionTok{theme\_light}\NormalTok{()}

\NormalTok{before\_gamma\_plot}\OtherTok{\textless{}{-}}\FunctionTok{ggplot}\NormalTok{(}\FunctionTok{aes}\NormalTok{(}\AttributeTok{x=}\NormalTok{accepted\_samples\_before[,}\DecValTok{2}\NormalTok{]), }\AttributeTok{data =} \FunctionTok{as.data.frame}\NormalTok{(accepted\_samples\_before))}\SpecialCharTok{+}
  \FunctionTok{geom\_density}\NormalTok{(}\AttributeTok{fill=}\StringTok{"pink"}\NormalTok{,}\AttributeTok{color=}\StringTok{"red"}\NormalTok{)}\SpecialCharTok{+}
  \FunctionTok{xlim}\NormalTok{(}\FunctionTok{c}\NormalTok{(}\DecValTok{0}\NormalTok{,}\DecValTok{1}\NormalTok{))}\SpecialCharTok{+}
  \FunctionTok{labs}\NormalTok{(}\AttributeTok{x=}\StringTok{"Gamma"}\NormalTok{,}\AttributeTok{y=}\StringTok{"Density"}\NormalTok{,}\AttributeTok{title=}\FunctionTok{expression}\NormalTok{(}\FunctionTok{paste}\NormalTok{(}\StringTok{"Distribution of "}\NormalTok{, gamma,}\StringTok{" (Before)"}\NormalTok{)))}\SpecialCharTok{+}\FunctionTok{theme\_light}\NormalTok{()}

\NormalTok{after\_beta\_plot}\OtherTok{\textless{}{-}}\FunctionTok{ggplot}\NormalTok{(}\FunctionTok{aes}\NormalTok{(}\AttributeTok{x=}\NormalTok{accepted\_samples\_after[,}\DecValTok{1}\NormalTok{]), }\AttributeTok{data =} \FunctionTok{as.data.frame}\NormalTok{(accepted\_samples\_after))}\SpecialCharTok{+}
  \FunctionTok{geom\_density}\NormalTok{(}\AttributeTok{fill=}\StringTok{"lightblue"}\NormalTok{,}\AttributeTok{color=}\StringTok{"blue"}\NormalTok{)}\SpecialCharTok{+}
  \FunctionTok{xlim}\NormalTok{(}\FunctionTok{c}\NormalTok{(}\DecValTok{0}\NormalTok{,}\DecValTok{3}\NormalTok{))}\SpecialCharTok{+}
  \FunctionTok{labs}\NormalTok{(}\AttributeTok{x=}\StringTok{"Beta"}\NormalTok{,}\AttributeTok{y=}\StringTok{"Frequency"}\NormalTok{,}\AttributeTok{title=}\FunctionTok{expression}\NormalTok{(}\FunctionTok{paste}\NormalTok{(}\StringTok{"Distribution of "}\NormalTok{, beta, }\StringTok{" (After)"}\NormalTok{)))}\SpecialCharTok{+}\FunctionTok{theme\_light}\NormalTok{()}

\NormalTok{after\_gamma\_plot}\OtherTok{\textless{}{-}}\FunctionTok{ggplot}\NormalTok{(}\FunctionTok{aes}\NormalTok{(}\AttributeTok{x=}\NormalTok{accepted\_samples\_after[,}\DecValTok{2}\NormalTok{]), }\AttributeTok{data =} \FunctionTok{as.data.frame}\NormalTok{(accepted\_samples\_after))}\SpecialCharTok{+}
  \FunctionTok{geom\_density}\NormalTok{(}\AttributeTok{fill=}\StringTok{"pink"}\NormalTok{,}\AttributeTok{color=}\StringTok{"red"}\NormalTok{)}\SpecialCharTok{+}
  \FunctionTok{xlim}\NormalTok{(}\FunctionTok{c}\NormalTok{(}\DecValTok{0}\NormalTok{,}\DecValTok{1}\NormalTok{))}\SpecialCharTok{+}
  \FunctionTok{labs}\NormalTok{(}\AttributeTok{x=}\StringTok{"Gamma"}\NormalTok{,}\AttributeTok{y=}\StringTok{"Density"}\NormalTok{,}\AttributeTok{title=}\FunctionTok{expression}\NormalTok{(}\FunctionTok{paste}\NormalTok{(}\StringTok{"Distribution of "}\NormalTok{, gamma, }\StringTok{" (After)"}\NormalTok{)))}\SpecialCharTok{+}\FunctionTok{theme\_light}\NormalTok{()}

\FunctionTok{plot\_grid}\NormalTok{(before\_beta\_plot,after\_beta\_plot,before\_gamma\_plot,after\_gamma\_plot,}\AttributeTok{nrow=}\DecValTok{2}\NormalTok{,}\AttributeTok{ncol=}\DecValTok{2}\NormalTok{)}
\end{Highlighting}
\end{Shaded}

\includegraphics{Project_files/figure-latex/unnamed-chunk-15-1.pdf}

Indeed, as we decrease the value of \(\epsilon\), the ABC-estimated
posterior distribution of \(\beta\) and \(\gamma\) diverges further and
further from the uniform priors \(U[0,3]\), \(U[0,1]\), and converges
towards a distribution centered around the MAP (maximum a priori)
estimate.

We can see this in the plot below, which begins with a tolerance value
of 0.4, producing a wider plot, and with each decrease in \(\epsilon\),
more and more mass is shifted towards the `true' value. Whilst not
included here, the same is true for \(\gamma\) also.

\begin{Shaded}
\begin{Highlighting}[]
\NormalTok{epsilons }\OtherTok{\textless{}{-}} \FunctionTok{c}\NormalTok{(}\FloatTok{0.5}\NormalTok{,}\FloatTok{0.4}\NormalTok{,}\FloatTok{0.3}\NormalTok{,}\FloatTok{0.2}\NormalTok{)}
\NormalTok{datas }\OtherTok{\textless{}{-}} \FunctionTok{list}\NormalTok{(}\FunctionTok{c}\NormalTok{(),}\FunctionTok{c}\NormalTok{(),}\FunctionTok{c}\NormalTok{(),}\FunctionTok{c}\NormalTok{())}
\ControlFlowTok{for}\NormalTok{ (i }\ControlFlowTok{in} \DecValTok{1}\SpecialCharTok{:}\DecValTok{4}\NormalTok{)\{}
\NormalTok{  sims }\OtherTok{\textless{}{-}} \FunctionTok{ABC}\NormalTok{(}\FloatTok{1e6}\NormalTok{, epsilons[i], }\DecValTok{2}\NormalTok{, x, }\DecValTok{4}\NormalTok{, }\DecValTok{1}\NormalTok{,}\ConstantTok{NA}\NormalTok{)}
\NormalTok{  datas[[i]] }\OtherTok{\textless{}{-}}\NormalTok{ sims[}\SpecialCharTok{!}\FunctionTok{rowSums}\NormalTok{(sims}\SpecialCharTok{==}\DecValTok{0}\NormalTok{),][,}\DecValTok{1}\NormalTok{]}
\NormalTok{\}}

\FunctionTok{ggplot}\NormalTok{(}\FunctionTok{aes}\NormalTok{(}\AttributeTok{x=}\NormalTok{datas[[}\DecValTok{1}\NormalTok{]]),}\AttributeTok{data=}\FunctionTok{as.data.frame}\NormalTok{(datas[[}\DecValTok{1}\NormalTok{]]))}\SpecialCharTok{+}
  \FunctionTok{geom\_density}\NormalTok{(}\AttributeTok{fill=}\StringTok{\textquotesingle{}blue\textquotesingle{}}\NormalTok{,}\AttributeTok{alpha=}\NormalTok{.}\DecValTok{2}\NormalTok{)}\SpecialCharTok{+}
  \FunctionTok{geom\_density}\NormalTok{(}\AttributeTok{data=}\FunctionTok{as.data.frame}\NormalTok{(datas[[}\DecValTok{2}\NormalTok{]]),}\FunctionTok{aes}\NormalTok{(}\AttributeTok{x=}\NormalTok{datas[[}\DecValTok{2}\NormalTok{]]),}\AttributeTok{fill=}\StringTok{\textquotesingle{}blue\textquotesingle{}}\NormalTok{,}\AttributeTok{alpha=}\NormalTok{.}\DecValTok{2}\NormalTok{)}\SpecialCharTok{+}
  \FunctionTok{geom\_density}\NormalTok{(}\AttributeTok{data=}\FunctionTok{as.data.frame}\NormalTok{(datas[[}\DecValTok{3}\NormalTok{]]),}\FunctionTok{aes}\NormalTok{(}\AttributeTok{x=}\NormalTok{datas[[}\DecValTok{3}\NormalTok{]]),}\AttributeTok{fill=}\StringTok{\textquotesingle{}blue\textquotesingle{}}\NormalTok{,}\AttributeTok{alpha=}\NormalTok{.}\DecValTok{2}\NormalTok{)}\SpecialCharTok{+}
  \FunctionTok{geom\_density}\NormalTok{(}\AttributeTok{data=}\FunctionTok{as.data.frame}\NormalTok{(datas[[}\DecValTok{4}\NormalTok{]]),}\FunctionTok{aes}\NormalTok{(}\AttributeTok{x=}\NormalTok{datas[[}\DecValTok{4}\NormalTok{]]),}\AttributeTok{fill=}\StringTok{\textquotesingle{}blue\textquotesingle{}}\NormalTok{,}\AttributeTok{alpha=}\NormalTok{.}\DecValTok{2}\NormalTok{)}\SpecialCharTok{+}
  \FunctionTok{xlim}\NormalTok{(}\FunctionTok{c}\NormalTok{(}\DecValTok{0}\NormalTok{,}\DecValTok{3}\NormalTok{))}\SpecialCharTok{+}
  \FunctionTok{labs}\NormalTok{(}\AttributeTok{x=}\StringTok{"Beta"}\NormalTok{,}\AttributeTok{y=}\StringTok{"Frequency"}\NormalTok{,}\AttributeTok{title=}\FunctionTok{expression}\NormalTok{(}\FunctionTok{paste}\NormalTok{(}\StringTok{"Distribution of "}\NormalTok{, beta,}\StringTok{"  as  "}\NormalTok{, epsilon, }\StringTok{" approaches 0.1"}\NormalTok{)))}\SpecialCharTok{+}\FunctionTok{theme\_light}\NormalTok{()}
\end{Highlighting}
\end{Shaded}

\includegraphics{Project_files/figure-latex/unnamed-chunk-16-1.pdf}

\subsection{Parameter Estimates}\label{parameter-estimates}

We will now simulate the SIR model using the optimal values of \(\beta\)
and \(\gamma\). To best gain these optimal point estimates from our
posterior distributions produced by ABC, we use a method of kernel
density estimation.

Below is a function which calculates these estimates for \(\hat{\beta}\)
and \(\hat{\gamma}\), then is able to plot the associated heatmap used
to find these posterior maximisers, as well as the fit to the observed
data produced when inputting these estimates into the SIR simulator.

\begin{Shaded}
\begin{Highlighting}[]
\FunctionTok{library}\NormalTok{(MASS)}
\FunctionTok{library}\NormalTok{(ks)}

\NormalTok{Optimal\_Plot}\OtherTok{\textless{}{-}}\ControlFlowTok{function}\NormalTok{(sims,plot.heat)\{}
  
  \FunctionTok{colnames}\NormalTok{(sims) }\OtherTok{\textless{}{-}} \FunctionTok{c}\NormalTok{(}\StringTok{"beta"}\NormalTok{, }\StringTok{"gamma"}\NormalTok{)}
  
\NormalTok{  kde\_result }\OtherTok{\textless{}{-}} \FunctionTok{kde2d}\NormalTok{(sims[,}\DecValTok{1}\NormalTok{], sims[,}\DecValTok{2}\NormalTok{], }\AttributeTok{n=}\DecValTok{50}\NormalTok{)}
  
\NormalTok{  kde\_df }\OtherTok{\textless{}{-}} \FunctionTok{data.frame}\NormalTok{(}
  \AttributeTok{x =} \FunctionTok{rep}\NormalTok{(kde\_result}\SpecialCharTok{$}\NormalTok{x, }\AttributeTok{each =} \FunctionTok{length}\NormalTok{(kde\_result}\SpecialCharTok{$}\NormalTok{y)),}
  \AttributeTok{y =} \FunctionTok{rep}\NormalTok{(kde\_result}\SpecialCharTok{$}\NormalTok{y, }\AttributeTok{times =} \FunctionTok{length}\NormalTok{(kde\_result}\SpecialCharTok{$}\NormalTok{x)),}
  \AttributeTok{z =} \FunctionTok{as.vector}\NormalTok{(kde\_result}\SpecialCharTok{$}\NormalTok{z)}
\NormalTok{  )}
  
\NormalTok{  heat\_plot }\OtherTok{\textless{}{-}} \FunctionTok{ggplot}\NormalTok{(kde\_df, }\FunctionTok{aes}\NormalTok{(}\AttributeTok{x =}\NormalTok{ x, }\AttributeTok{y =}\NormalTok{ y, }\AttributeTok{z =}\NormalTok{ z)) }\SpecialCharTok{+}
  \FunctionTok{geom\_contour\_filled}\NormalTok{() }\SpecialCharTok{+}
  \FunctionTok{labs}\NormalTok{(}\AttributeTok{title =} \StringTok{"2D KDE Filled Contour Plot"}\NormalTok{,}
       \AttributeTok{x =} \StringTok{"X{-}axis"}\NormalTok{,}
       \AttributeTok{y =} \StringTok{"Y{-}axis"}\NormalTok{,}
       \AttributeTok{fill =} \StringTok{"Density"}\NormalTok{) }\SpecialCharTok{+}
  \FunctionTok{theme\_minimal}\NormalTok{()}
  
  \CommentTok{\# Extract KDE results}
\NormalTok{  x }\OtherTok{\textless{}{-}}\NormalTok{ kde\_result}\SpecialCharTok{$}\NormalTok{x}
\NormalTok{  y }\OtherTok{\textless{}{-}}\NormalTok{ kde\_result}\SpecialCharTok{$}\NormalTok{y}
\NormalTok{  z }\OtherTok{\textless{}{-}}\NormalTok{ kde\_result}\SpecialCharTok{$}\NormalTok{z}
  
  \CommentTok{\# Find the indices of the maximum density value}
\NormalTok{  max\_density\_index }\OtherTok{\textless{}{-}} \FunctionTok{which}\NormalTok{(z }\SpecialCharTok{==} \FunctionTok{max}\NormalTok{(z), }\AttributeTok{arr.ind =} \ConstantTok{TRUE}\NormalTok{)}
  
  \CommentTok{\# Get the corresponding x and y values for the maximum density}
\NormalTok{  max\_x }\OtherTok{\textless{}{-}}\NormalTok{ x[max\_density\_index[}\DecValTok{1}\NormalTok{]]}
\NormalTok{  max\_y }\OtherTok{\textless{}{-}}\NormalTok{ y[max\_density\_index[}\DecValTok{2}\NormalTok{]]}
  
\NormalTok{  I }\OtherTok{\textless{}{-}} \FunctionTok{SIR}\NormalTok{(}\FunctionTok{numeric}\NormalTok{(}\DecValTok{20}\NormalTok{), }\FunctionTok{numeric}\NormalTok{(}\DecValTok{20}\NormalTok{), }\FunctionTok{numeric}\NormalTok{(}\DecValTok{20}\NormalTok{), }\DecValTok{1} \SpecialCharTok{{-}} \DecValTok{1}\SpecialCharTok{/}\DecValTok{763}\NormalTok{, }\DecValTok{1}\SpecialCharTok{/}\DecValTok{763}\NormalTok{, }\DecValTok{0}\NormalTok{, max\_x, max\_y)}
  
\NormalTok{  obs\_data }\OtherTok{\textless{}{-}} \FunctionTok{c}\NormalTok{(}\FunctionTok{rep}\NormalTok{(}\DecValTok{1}\SpecialCharTok{/}\DecValTok{763}\NormalTok{,}\DecValTok{5}\NormalTok{), }\DecValTok{2}\SpecialCharTok{/}\DecValTok{763}\NormalTok{, bsflu}\SpecialCharTok{$}\NormalTok{B}\SpecialCharTok{/}\DecValTok{763}\NormalTok{)}
  
\NormalTok{  fit\_plot }\OtherTok{\textless{}{-}} \FunctionTok{ggplot}\NormalTok{(}\AttributeTok{data=}\FunctionTok{data.frame}\NormalTok{(}\AttributeTok{z=}\FunctionTok{c}\NormalTok{(}\FunctionTok{seq}\NormalTok{(}\DecValTok{1}\NormalTok{,}\DecValTok{20}\NormalTok{),}\FunctionTok{seq}\NormalTok{(}\DecValTok{1}\NormalTok{,}\DecValTok{20}\NormalTok{)),}\AttributeTok{y=}\FunctionTok{c}\NormalTok{(obs\_data,I),}\AttributeTok{Data=}\FunctionTok{c}\NormalTok{(}\FunctionTok{rep}\NormalTok{(}\StringTok{"Observed"}\NormalTok{,}\DecValTok{20}\NormalTok{  ),}\FunctionTok{rep}\NormalTok{(}\StringTok{"Simulation"}\NormalTok{,}\DecValTok{20}\NormalTok{))),}\FunctionTok{aes}\NormalTok{(}\AttributeTok{x=}\NormalTok{z,}\AttributeTok{y=}\NormalTok{y,}\AttributeTok{col=}\NormalTok{Data)) }\SpecialCharTok{+}
    \FunctionTok{geom\_point}\NormalTok{() }\SpecialCharTok{+} \FunctionTok{geom\_line}\NormalTok{() }\SpecialCharTok{+} \FunctionTok{xlab}\NormalTok{(}\StringTok{"Day"}\NormalTok{) }\SpecialCharTok{+} \FunctionTok{ylab}\NormalTok{(}\StringTok{"Proportion of Infected"}\NormalTok{) }\SpecialCharTok{+}     \FunctionTok{ggtitle}\NormalTok{(}\StringTok{"ABC Estimated Fit vs. Observed Data"}\NormalTok{)}
  
  \ControlFlowTok{if}\NormalTok{(plot.heat)\{}
    
    \FunctionTok{plot\_grid}\NormalTok{(heat\_plot,fit\_plot)}
    
\NormalTok{  \}}
  \ControlFlowTok{else}\NormalTok{\{}
    
    \FunctionTok{return}\NormalTok{(fit\_plot)}
    
\NormalTok{  \}}
\NormalTok{\}}
\end{Highlighting}
\end{Shaded}

As an example therefore, we can use the mean squared error distance
metric, along with a very small tolerance value of 0.01 to produce this
fit to the data.

\begin{Shaded}
\begin{Highlighting}[]
\NormalTok{accepted\_samples }\OtherTok{\textless{}{-}} \FunctionTok{ABC}\NormalTok{(}\FloatTok{1e6}\NormalTok{, .}\DecValTok{1}\NormalTok{, }\DecValTok{2}\NormalTok{, x, }\DecValTok{4}\NormalTok{, }\DecValTok{1}\NormalTok{,}\ConstantTok{NA}\NormalTok{)}
\NormalTok{accepted\_samples }\OtherTok{\textless{}{-}}\NormalTok{ accepted\_samples[}\SpecialCharTok{!}\FunctionTok{rowSums}\NormalTok{(accepted\_samples}\SpecialCharTok{==}\DecValTok{0}\NormalTok{),]}


\FunctionTok{Optimal\_Plot}\NormalTok{(accepted\_samples,}\ConstantTok{TRUE}\NormalTok{)}
\end{Highlighting}
\end{Shaded}

\includegraphics{Project_files/figure-latex/unnamed-chunk-18-1.pdf}

By observing the contour plot, we can see that the optimal parameters
using this method is about \(\hat{\gamma} = 0.45\), and
\(1.2 < \hat{\beta} < 1.3\). Then by plugging these into the SIR
simulator, we get a simulation that matches the observed data well.
However it is worth noting that to achieve such a good fit required an
incredibly low acceptance rate: only 0.86\%.

\subsection{Extension: Summary
Statistics}\label{extension-summary-statistics}

One other distance metric we could have used was that of looking at the
difference between a handful of \emph{summary statistics} of the data,
rather than the raw data itself. This reduces the dimensionality of the
problem and can reduce the amount of simulations needed for a good fit.

The example we use here looks at the empirical mean and variance of the
distribution, as well as the `peak' of the distribution. I.e. for
observed data points \((x_i,y_i)\) where \(x_i\) is normalised to be a
probability,

\[
\mu = \sum_{i=1}^{N}x_i y_i
\]

\[
\sigma^2 = \sum_{i=1}^N y_i(\mu-x_i)^2
\]

Then by observing \(d_{obs} = (x_{obs},y_{obs})\) and simulating
\(d_{sim} = (x_{sim},y_{sim})\), we define our distance metric to be:

\[
D(d_{obs},d_{sim}) = w_1\frac{(\mu_{obs}-\mu_{sim})^2}{\mu_{obs}^2} + w_2\frac{(\sigma_{obs}-\sigma_{sim})^2}{\sigma_{obs}^2} + w_3\frac{(max(d_{obs})-max(d_{sim}))^2}{max(d_{obs})^2}
\]

Where \(w_1, w_2, w_3\) are the weights assigned to each distance
measure.

Then by using a different set of weights, we can assign more importance
to the matching of mean, variance, and whether the peaks of the
distributions coincide. For example, by prioritizing each we obtain the
plots below, where the mean weighted one matches the mean perfectly, but
at the expense of standard deviation. The estimate weighted by peak
however matches the peak and shape of distribution correctly, but with
the wrong mean.

\begin{Shaded}
\begin{Highlighting}[]
\NormalTok{weights }\OtherTok{\textless{}{-}} \FunctionTok{list}\NormalTok{(}\FunctionTok{c}\NormalTok{(}\DecValTok{2}\NormalTok{,.}\DecValTok{2}\NormalTok{,.}\DecValTok{2}\NormalTok{),}\FunctionTok{c}\NormalTok{(.}\DecValTok{2}\NormalTok{,}\DecValTok{1}\NormalTok{,.}\DecValTok{2}\NormalTok{),}\FunctionTok{c}\NormalTok{(.}\DecValTok{2}\NormalTok{,.}\DecValTok{2}\NormalTok{,}\DecValTok{2}\NormalTok{))}
\NormalTok{plots }\OtherTok{\textless{}{-}} \FunctionTok{list}\NormalTok{(before\_gamma\_plot,before\_beta\_plot,before\_beta\_plot)}

\ControlFlowTok{for}\NormalTok{ (i }\ControlFlowTok{in} \DecValTok{1}\SpecialCharTok{:}\DecValTok{3}\NormalTok{)\{}
  
\NormalTok{  accepted\_samples }\OtherTok{\textless{}{-}} \FunctionTok{ABC}\NormalTok{(}\FloatTok{1e6}\NormalTok{, .}\DecValTok{4}\NormalTok{, }\DecValTok{2}\NormalTok{, x, }\DecValTok{4}\NormalTok{, }\DecValTok{3}\NormalTok{, weights[[i]])}
\NormalTok{  accepted\_samples }\OtherTok{\textless{}{-}}\NormalTok{ accepted\_samples[}\SpecialCharTok{!}\FunctionTok{rowSums}\NormalTok{(accepted\_samples}\SpecialCharTok{==}\DecValTok{0}\NormalTok{),]}
\NormalTok{  plots[[i]] }\OtherTok{\textless{}{-}} \FunctionTok{Optimal\_Plot}\NormalTok{(accepted\_samples,}\ConstantTok{FALSE}\NormalTok{)}
  
\NormalTok{\}}

\FunctionTok{plot\_grid}\NormalTok{(plots[[}\DecValTok{1}\NormalTok{]],plots[[}\DecValTok{2}\NormalTok{]],plots[[}\DecValTok{3}\NormalTok{]])}
\end{Highlighting}
\end{Shaded}

\includegraphics{Project_files/figure-latex/unnamed-chunk-19-1.pdf}

Then by choosing an appropriate set of weights and a low value of
\(\epsilon\), we can achieve a fit to the observed data of similar
quality to that which we got using the mean squared error of raw data.
The difference here is the acceptance probability, while low, is much
higher than that case: 4\% as opposed to 0.8\%. Hence requiring less
computational power due to less simulations required.

\begin{Shaded}
\begin{Highlighting}[]
\NormalTok{accepted\_samples }\OtherTok{\textless{}{-}} \FunctionTok{ABC}\NormalTok{(}\FloatTok{1e6}\NormalTok{, .}\DecValTok{05}\NormalTok{, }\DecValTok{2}\NormalTok{, x, }\DecValTok{4}\NormalTok{, }\DecValTok{3}\NormalTok{, }\FunctionTok{c}\NormalTok{(}\DecValTok{1}\NormalTok{,.}\DecValTok{1}\NormalTok{,.}\DecValTok{5}\NormalTok{))}
\NormalTok{accepted\_samples }\OtherTok{\textless{}{-}}\NormalTok{ accepted\_samples[}\SpecialCharTok{!}\FunctionTok{rowSums}\NormalTok{(accepted\_samples}\SpecialCharTok{==}\DecValTok{0}\NormalTok{),]}
\FunctionTok{Optimal\_Plot}\NormalTok{(accepted\_samples,}\ConstantTok{FALSE}\NormalTok{)}
\end{Highlighting}
\end{Shaded}

\includegraphics{Project_files/figure-latex/unnamed-chunk-20-1.pdf}

\subsection{References}\label{references}

{[}1{]} J.-M. Marin, P. Pudlo, C. P. Robert, and R. J. Ryder,
``Approximate bayesian computational methods'' Stat Comput, vol.~22,
pp.~1167--1180, Nov.~2012.

{[}2{]} A. Minter, and R. Retkute, ``Approximate Bayesian Computation
for infectious disease modelling''. Epidemics, vol.~29, p.100368.

\end{document}
